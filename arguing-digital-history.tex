\documentclass[11pt,]{article}
\usepackage[left=1in,top=1in,right=1in,bottom=1in]{geometry}
\newcommand*{\authorfont}{\fontfamily{phv}\selectfont}
\usepackage[]{mathpazo}


  \usepackage[T1]{fontenc}
  \usepackage[utf8]{inputenc}



\usepackage{abstract}
\renewcommand{\abstractname}{}    % clear the title
\renewcommand{\absnamepos}{empty} % originally center

\renewenvironment{abstract}
 {{%
    \setlength{\leftmargin}{0mm}
    \setlength{\rightmargin}{\leftmargin}%
  }%
  \relax}
 {\endlist}

\makeatletter
\def\@maketitle{%
  \newpage
%  \null
%  \vskip 2em%
%  \begin{center}%
  \let \footnote \thanks
    {\fontsize{18}{20}\selectfont\raggedright  \setlength{\parindent}{0pt} \@title \par}%
}
%\fi
\makeatother




\setcounter{secnumdepth}{0}



\title{Arguing with Digital History Position Paper  }



\author{\Large Jason A. Heppler, Ph.D.\vspace{0.05in} \newline\normalsize\emph{University of Nebraska at Omaha}  }


\date{}

\usepackage{titlesec}

\titleformat*{\section}{\normalsize\bfseries}
\titleformat*{\subsection}{\normalsize\itshape}
\titleformat*{\subsubsection}{\normalsize\itshape}
\titleformat*{\paragraph}{\normalsize\itshape}
\titleformat*{\subparagraph}{\normalsize\itshape}


\usepackage{natbib}
\bibliographystyle{plainnat}



\newtheorem{hypothesis}{Hypothesis}
\usepackage{setspace}

\makeatletter
\@ifpackageloaded{hyperref}{}{%
\ifxetex
  \PassOptionsToPackage{hyphens}{url}\usepackage[setpagesize=false, % page size defined by xetex
              unicode=false, % unicode breaks when used with xetex
              xetex]{hyperref}
\else
  \PassOptionsToPackage{hyphens}{url}\usepackage[unicode=true]{hyperref}
\fi
}

\@ifpackageloaded{color}{
    \PassOptionsToPackage{usenames,dvipsnames}{color}
}{%
    \usepackage[usenames,dvipsnames]{color}
}
\makeatother
\hypersetup{breaklinks=true,
            bookmarks=true,
            pdfauthor={Jason A. Heppler, Ph.D. (University of Nebraska at Omaha)},
             pdfkeywords = {digital history, digital humanities, narrative, historical theory,
historical argumentation},  
            pdftitle={Arguing with Digital History Position Paper},
            colorlinks=true,
            citecolor=blue,
            urlcolor=blue,
            linkcolor=magenta,
            pdfborder={0 0 0}}
\urlstyle{same}  % don't use monospace font for urls



% add tightlist ----------
\providecommand{\tightlist}{%
\setlength{\itemsep}{0pt}\setlength{\parskip}{0pt}}

\begin{document}
	
% \pagenumbering{arabic}% resets `page` counter to 1 
%
% \maketitle

{% \usefont{T1}{pnc}{m}{n}
\setlength{\parindent}{0pt}
\thispagestyle{plain}
{\fontsize{18}{20}\selectfont\raggedright 
\maketitle  % title \par  

}

{
   \vskip 13.5pt\relax \normalsize\fontsize{11}{12} 
\textbf{\authorfont Jason A. Heppler, Ph.D.} \hskip 15pt \emph{\small University of Nebraska at Omaha}   

}

}







\begin{abstract}

    \hbox{\vrule height .2pt width 39.14pc}

    \vskip 8.5pt % \small 

\noindent Prepared for the \emph{Arguing with Digital History} workshop at George
Mason University, September 14-16, 2017.


\vskip 8.5pt \noindent \emph{Keywords}: digital history, digital humanities, narrative, historical theory,
historical argumentation \par

    \hbox{\vrule height .2pt width 39.14pc}



\end{abstract}


\vskip 6.5pt

\noindent  First, a loose definition of ``digital history'': I take digital history
to mean a variety of approaches to using computational, visual, and
informational methods in analyzing, visualizing, and presenting
historical analysis and arguments. As I consider the ways that ``digital
history'' can engage with argumentation, then, I assume particularities
of digital history that rely on graphical displays (maps, networks),
take advantage of narrative and hypertextuality, provide access to
digitized primary sources, share historical evidence and data compiled
for computation and visualization, and that takes the form of a digital
scholarly website that integrates some or all of these aspects of
research and communication. Such projects are distinct from efforts that
define themselves as public history; while much of digital history can
be said to be public scholarship, that does not mean they rise to the
criteria of public history.\footnote{Jason Heppler, et al., ``Public
  History as Digital History as Public History,'' working group,
  National Council on Public History Annual Meeting, Nashville,
  Tennessee, April 2015.} Under this definition, to state it plainly,
digital history must be explicitly \emph{digital}: that the arguments,
visualizations, and narrative are much harder to achieve in print and,
thus, can only exist as a digital version.

Many digital history projects tend to fall into the realms of digital
collections or data visualization. Some of these projects have led to
more traditional venues of historical writing and publishing. The
\emph{Valley of the Shadow}, an archive of Civil War-era material
detailing the experiences of Confederate and Union soldiers in Augusta
and Franklin Counties in Virginia, led to the publication of ``The
Differences Slavery Made'' by Ed Ayers and William G. Thomas, as well as
Ayers' \emph{In the Presence of Mine Enemies}.\footnote{\emph{The Valley
  of the Shadow: Two Communities in the American Civil War}, University
  of Virginia \url{http://valley.lib.virginia.edu/}; Edward L. Ayres,
  \emph{In the Presence of Mine Enemies: The Civil War in the Heart of
  America, 1859---1864} (New York: Norton, 2004).} The appearance of
digital history projects alongside print continues: Richard White's
\emph{Railroaded} was accompanied by a digital component\footnote{Richard
  White, \emph{Railroaded: The Transcontinentals and the Making of
  Modern America} (New York: Norton, 2012). White's digital companion,
  \emph{Railroaded}, is found at \url{http://railroaded.stanford.edu}.};
William G. Thomas's \emph{The Iron Way} likewise had a digital
component, providing access to primary sources, data, and
visualizations\footnote{William G. Thomas, \emph{The Iron Way:
  Railroads, the Civil War, and the Making of Modern America} (New
  Haven: Yale University Press, 2011). Thomas's digital companion,
  \emph{Railroads and the Making of Modern America}, is at
  \url{http://railroads.unl.edu/}.}; Karl Jacoby's \emph{Shadows at
Dawn} relies on a digital companion to provide access to primary
sources.\footnote{Karl Jacoby, \emph{Shadows at Dawn: An Apache Massacre
  and the Violence of History} (New York: Penguin Books, 2009). Jacoby's
  digital companion, \emph{Shadows at Dawn}, is at
  \url{http://brown.edu/Research/Aravaipa/}.} Conspicuously, these
digital components don't exist primarily for driving a historical
thesis, nor do they offer much in the way of expanding on narrative
elements of their print partners (or experiments in nonlinear
hypertextual narrative). They instead hinge on providing access to a set
of material used by the writers in the creation of their
monographs.\footnote{The same can be said for scholarly articles. Some
  digital projects provide digital-only whitepapers or pre-prints, such
  as Stanford's Spatial History Project
  \url{http://web.stanford.edu/group/spatialhistory/cgi-bin/site/pub_toc.php}.
  Other digital projects may be accompanied by print publications that
  appear alongside the digital. See, for example, the work of Claire
  Arcenas and Caroline Winterer, \emph{The Correspondence Network of
  Benjamin Franklin: The London Decades}
  \url{http://republicofletters.stanford.edu/publications/franklin/}.}

The list of digital history projects that attempt to exist as
digital-only publications is quite short, a situation that likely
reflects the promotion and tenure realities of the academy. Few history
departments have attempted to devise promotion structures and guidelines
that accommodate digital scholarship, instead continuing to value print
(and a book, in particular) as the cornerstone of scholarly
merit.\footnote{This is not to suggest there are no digital projects
  that attempt to be digital-only, but few of those have been peer
  reviewed let alone granted the imprimatur of an academic press. For
  example, see Douglas Seefeldt, \emph{Horrible Massacre of Emigrants!!
  The Mountain Meadows Massacre in Public Discourse}
  \url{http://mountainmeadows.unl.edu/}; Nick Bauch, \emph{Enchanting
  the Desert} \url{http://enchantingthedesert.com/}; Michelle Delaney
  and Rebecca Wingo, \emph{``I Shall Be Glad To See Them'': Gertrude
  Käsebier's ``Show Indian'' Photographs}
  \url{http://codystudies.org/kasebier/index.html}.} This is not,
however, a limitation of how digital history can engage with
argumentation. Digital history is no different, in some ways, with more
traditional approaches to history: an interpretation built upon evidence
and that recognizes context. While digital history can change how
argumentation might look, by being more computational, more visual, and
rely on large-scale machine-readable sources, argumentation otherwise
utilizes the same techniques and venues for contributing to historical
knowledge. While these changes in digital historical argumentation
influence the kinds of questions we can ask---after all, only a computer
can read and find patterns in 100,000 pages of newspapers\footnote{Library
  of Congress \emph{Chronicling America}
  \url{http://chroniclingamerica.loc.gov/about/}}---the answers must
still grapple with context and existing historical interpretation.

Insofar, then, that any notion of ``digital history'' supposes some
distinction from ``traditional'' history, the notion mostly comes down
to \emph{form}. There are cases, such as \emph{Mapping the Republic of
Letters}, where digital projects do not make historical claims. But that
presupposes the purpose of such projects. These projects exist not to
make arguments in and of themselves, but as \emph{aids} to research and
new knowledge that result in publications destined for traditional
publishing venues---and, for better or worse, where such work reaches a
specific audience. In other words, these projects still strive for
argumentation even though their eventual form may not be only digital.

In my own work, I have pursued projects that exist as both print and
digital with the exception of one project. The first of these projects,
\emph{Framing Red Power: The American Indian Movement, the Trail of
Broken Treaties, and the Politics of Media} served as a digital
companion to my Masters thesis and contained narrative, primary sources,
and data visualization.\footnote{\emph{Framing Red Power: The American
  Indian Movement, the Trail of Broken Treaties, and the Politics of
  Media} \url{http://framingredpower.org}.} That project served as an
aid to research, but also exists as a stand-alone project containing a
greatly abridged version of the traditional thesis. My second project,
still in progress, called \emph{``Self-sustaining and a good citizen'':
William F. Cody and the Progressive Wild West}, stands as a digital-only
publication that will be going up for peer review.\footnote{\emph{``Self-sustaining
  and a good citizen'': William F. Cody and the Progressive Wild West}
  \url{http://codystudies.org}.} No print version of this project will
exist, but will be accompanied by the imprimatur of a scholarly
authority. My third project, also in progress, is \emph{Machines in the
Valley}, which served as a digital companion to my dissertation but is
being expanded upon as I begin working on my book manuscript.\footnote{\emph{Machines
  in the Valley: Growth, Conflict, and Environmental Politics in Silicon
  Valley} \url{http://dissertation.jasonheppler.org}.} In each of these
cases, however, I have tried to design the projects to stand on their
own---to engage with a scholarly literature, to narrate historical
events, and to take advantage of computational methods to aid
interpretation. My projects have tended to serve as
publically-accessible digital scholarship: to provide access to
historical sources, but also contextualize and narrate those sources
with the hope that others may build off that work.


\newpage
\singlespacing 
\end{document}
