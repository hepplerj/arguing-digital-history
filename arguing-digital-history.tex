\documentclass[11pt,]{article}
\usepackage[left=1in,top=1in,right=1in,bottom=1in]{geometry}
\newcommand*{\authorfont}{\fontfamily{phv}\selectfont}
\usepackage[]{mathpazo}


  \usepackage[T1]{fontenc}
  \usepackage[utf8]{inputenc}



\usepackage{abstract}
\renewcommand{\abstractname}{}    % clear the title
\renewcommand{\absnamepos}{empty} % originally center

\renewenvironment{abstract}
 {{%
    \setlength{\leftmargin}{0mm}
    \setlength{\rightmargin}{\leftmargin}%
  }%
  \relax}
 {\endlist}

\makeatletter
\def\@maketitle{%
  \newpage
%  \null
%  \vskip 2em%
%  \begin{center}%
  \let \footnote \thanks
    {\fontsize{18}{20}\selectfont\raggedright  \setlength{\parindent}{0pt} \@title \par}%
}
%\fi
\makeatother




\setcounter{secnumdepth}{0}



\title{Arguing with Digital History  }



\author{\Large Jason A. Heppler, Ph.D.\vspace{0.05in} \newline\normalsize\emph{University of Nebraska at Omaha}  }


\date{}

\usepackage{titlesec}

\titleformat*{\section}{\normalsize\bfseries}
\titleformat*{\subsection}{\normalsize\itshape}
\titleformat*{\subsubsection}{\normalsize\itshape}
\titleformat*{\paragraph}{\normalsize\itshape}
\titleformat*{\subparagraph}{\normalsize\itshape}


\usepackage{natbib}
\bibliographystyle{plainnat}



\newtheorem{hypothesis}{Hypothesis}
\usepackage{setspace}

\makeatletter
\@ifpackageloaded{hyperref}{}{%
\ifxetex
  \PassOptionsToPackage{hyphens}{url}\usepackage[setpagesize=false, % page size defined by xetex
              unicode=false, % unicode breaks when used with xetex
              xetex]{hyperref}
\else
  \PassOptionsToPackage{hyphens}{url}\usepackage[unicode=true]{hyperref}
\fi
}

\@ifpackageloaded{color}{
    \PassOptionsToPackage{usenames,dvipsnames}{color}
}{%
    \usepackage[usenames,dvipsnames]{color}
}
\makeatother
\hypersetup{breaklinks=true,
            bookmarks=true,
            pdfauthor={Jason A. Heppler, Ph.D. (University of Nebraska at Omaha)},
             pdfkeywords = {digital history, digital humanities, historical narrative, historical
theory, historical argumentation},  
            pdftitle={Arguing with Digital History},
            colorlinks=true,
            citecolor=blue,
            urlcolor=blue,
            linkcolor=magenta,
            pdfborder={0 0 0}}
\urlstyle{same}  % don't use monospace font for urls



% add tightlist ----------
\providecommand{\tightlist}{%
\setlength{\itemsep}{0pt}\setlength{\parskip}{0pt}}

\begin{document}
	
% \pagenumbering{arabic}% resets `page` counter to 1 
%
% \maketitle

{% \usefont{T1}{pnc}{m}{n}
\setlength{\parindent}{0pt}
\thispagestyle{plain}
{\fontsize{18}{20}\selectfont\raggedright 
\maketitle  % title \par  

}

{
   \vskip 13.5pt\relax \normalsize\fontsize{11}{12} 
\textbf{\authorfont Jason A. Heppler, Ph.D.} \hskip 15pt \emph{\small University of Nebraska at Omaha}   

}

}







\begin{abstract}

    \hbox{\vrule height .2pt width 39.14pc}

    \vskip 8.5pt % \small 

\noindent Prepared for the \emph{Arguing with Digital History} workshop at George
Mason University, September 14-16, 2017.


\vskip 8.5pt \noindent \emph{Keywords}: digital history, digital humanities, historical narrative, historical
theory, historical argumentation \par

    \hbox{\vrule height .2pt width 39.14pc}



\end{abstract}


\vskip 6.5pt

\noindent  Historical argumentation in digital history has fallen short, with
projects tending to fall in the realms of digital collections or data
visualization. Some of these projects have led to more traditional
venues of historical writing and publishing. The \emph{Valley of the
Shadow}, an archive of Civil War-era material detailing the experiences
of Confederate and Union soldiders in Augusta and Franklin Counties in
Virginia, led to the publication of ``The Differences Slavery Made'' by
Ed Ayers and William G. Thomas, as well as Ayers' \emph{In the Presence
of Mine Enemies: war in the Heart of America, 1859--1863}. The
appearance of digital history projects alongside print continues:
Richard White's \emph{Railroaded} was accompanied by a digital
component\footnote{Richard White, \emph{Railroaded: The
  Transcontinentals and the Making of Modern America} (New York:
  jNorton, 2012). White's digital companion, \emph{Railroaded}, is found
  at \url{http://railroaded.stanford.edu}.}; William G. Thomas's
\emph{The Iron Way} likewise had a digital component, providing access
to primary sources, data, and visualizations\footnote{William G. Thomas,
  \emph{The Iron Way: Railroads, the Civil War, and the Making of Modern
  America} (New Haven: Yale University Press, 2011). Thomas's digital
  companion, \emph{Railroads and the Making of Modern America}, is at
  \url{http://railroads.unl.edu/}.}; Karl Jacoby's \emph{Shadows at
Dawn} relies on a digital companion to provide access to primary
sources.\footnote{Karl Jacoby, \emph{Shadows at Dawn: An Apache Massacre
  and the Violence of History} (New York: Penguin Books, 2009). Jacoby's
  digital companion, \emph{Shadows at Dawn}, is at
  \url{http://brown.edu/Research/Aravaipa/}.} Conspicuously, the digital
components don't primarily exist for driving a historical thesis, nor do
they offer much in the way of expanding on narrative elements of their
print partners (or experiments in nonlinear hypertextual narrative).

Few digital history projects have attempted to exist as digital-only
publications.


\newpage
\singlespacing 
\end{document}
